\documentclass[10pt,fleqn,a4paper]{jsarticle}
\usepackage[dvipdfmx]{graphicx,color}
\usepackage{ascmac,amsmath,amssymb,amstext}
\usepackage{tikz,multicol,float,tikz-3dplot}
\usetikzlibrary{positioning,intersections,calc,arrows.meta,math,angles}
\usepackage{zogeny,ceo,comment}
\newcommand{\ans}[1]{\mbox{\boldmath{$#1$}}}
\renewcommand{\baselinestretch}{1.2} 
\tdplotsetmaincoords{60}{110}%{xy平面からどれだけ上にいるか(90度から-90度)}{z軸中心にどれだけ左右に回転するか}
\title{教育制度論:中間レポート-諸外国の教育制度について}
\author{応用化学科 16D0250057 大島 遙斗}
\begin{document}
\maketitle
\tableofcontents
\newpage
まず,私は韓国の教育制度について調べることにした.その経緯としては,韓国は我が国日本の隣国であり,十分に先進的な国
であるということである.そのとき,日本と比較して教育制度は優れているものと言えるのか否かや日本の学力と比較して日本より
劣っているの優れているのかについて気になったのである.
\section{韓国の学校体系の特徴について}
まず義務教育の期間は,日本と同じで初等学校が6年,中学校が3年の計9年間である.その後の進路も日本と似ていて,ほとんどすべて
の子供が,高等学校に進学する.これも日本と同様に教育期間は3年間である.一方,日本においても全日制の高等学校のほかに
工業専門学校などがあるように,韓国にも存在しており,一つ目に「特殊目的高校」,二つ目に「特性化高校」である.\\
 一つ目の特殊目的高校とは,「特殊分野の専門的な教育を目的とする高等学校」(初・中等教育法施行令第90条)であり,
一般的な高等学校と異なり,当該分野の特性に応じた入学者選抜方法,カリキュラム策定,評価方法で教育を行うことができる.加えて,
科学高校,外国語高校,芸術高校,体育高校の4つが特に才能教育機関とされている.また,全て公立なので手厚い支援を受けている.
この4つについては,セクション2で具体的に述べていく.\\
 二つ目の特性化高校とは,機械,電気,自動車,建築などの職業教育とされる学科で編成されており,各分野の才能と適性を備えた
学生に専門的な職業教育を行っている高校である.また,大学に進学する人以外は,2学期以降の成績はほとんど意味を持たないため,
韓国の強烈な受験競争には巻き込まれないという特徴がある.\\
 次に大学について述べていく.まず,大学に入学する方法(大学入試制度)であるが,次の3つがある.
$\maruichi 「大学修学能力試験」 \maruni 「学生総合型選抜」 ,\marusan 「論述思考」の3つである.\\
 \maruichi は「スヌン」と言われていて,
日本の大学入学共通テストに相当する試験で,その後の人生を決めるとも言われている試験である.これが日本でも度々紹介される
警察が送迎したりしている試験である.$また韓国の大学の教育期間は,日本と同じ学士が4年間,修士が2年間,博士が3年間である.\\
 $\maruni$は,日本で言う「総合方選抜」である.これは,学生簿を総合的に評価して学生を選抜する方法である.学生簿は,教師が作る
もので,その中には内申だけでなく,課外活動,資格,小論文,面接,自己紹介書など日本と同様,様々なことを考慮して選抜が行われている.\\
 $\marusan$は,スヌンが受験生全員が受験するのに対し,論述思考はスヌンのように点数が出るタイプではない.特に,SKY と呼ばれている
韓国内でトップとされている大学の入学試験に課されることが多い.



\newpage
\section{韓国の高校の実態について}
前のsection で述べた通り,韓国の高等学校は,一般的な高校に加え\tokeiichi 「特殊目的高校」と\tokeini 「特性化高校」がある.\\
 \tokeiichi「特殊目的高校(特に科学高校)」$\narabaa$ 物理,化学,生物などの科学技術に関連することや数学に関する教育に特化した学校である.
原則として全寮制を採用しており,1つの学級の人数は普通の公立高校が40人から50人程度であるのに対し,30人程度であり
手厚い教育が行われている. また,数学・科学関連のカリキュラムとして,「科学系列高等学校専門教育教育課程」が告示されて
いる.さらに、科学高校には,充実した施設・設備が用意されており,授業は日本で言う,探求活動に近い形,すなわち生徒同士の討論
や実験,実習を中心に進められている.記述した通り,科学高校は,非常に恵まれた教育環境を有しているのにも関わらず,学費
は一般の公立高等学校と同一になっている.\\
 科学高校の他にも,「外国語高校」,「国際高校」,「芸術高校」,「体育高校」,「マイスター高校」などが存在している.\\
 加えて,選抜方法も特徴がある.その前に「平準化」について述べておく.「平準化」とは,高校受験の競争率が社会問題になる中
高校の入試と教育環境を均等化し,教育格差と受験競争を減らすための制度である.この地域に指定されている釜山やソウルは競争試験に
よる選抜方法が一切認められていないが,特殊目的高である科学高校では競争試験による選抜が認められいる.\\
 \tokeini 「特性化高校」$\narabaa$ 前のsectionで述べた通り,大学進学を目標にするのではなく,高校卒業後に就職することを目的にしている
高校である.通常は,専門学科が編成されており,土木関連の学科や美容関係の学科など様々な学科が存在する.これも特殊目的高校と同様に,
1つの学級が20人から30人で構成されている.また,一般的な高校とさらに異なる点として,大学進学を目的としていないため,3年生の
夏休み以降(9月以降)は授業がほとんど行われない.その代わりに何が行われているかというと,就職のための面接の練習である.これが,2月まで
続く.\\
 カリキュラムの特徴もある.述べたとおり,就職することを目的にしているため,一般的な高校で実施されているような普通教育は実施されていない.
その代わりに,編成された学科の専門的な教育が展開されている.そのため,自分が興味のない学科に進学した場合,3年生のときの勉強がとても
辛いものになる場合もある.\\
 ここで先般,「特性化高校は,就職を目標にする高校である」と述べたが,この内更に高度な知識,技術を要する分野,「半導体,バイオ,量子,エネルギー」
などの分野は,特性化高校ではなく,特殊目的高校の内の「マイスター高校」で扱うことになっている.\\
 だが,特性化高校に対する世間の目は厳しい.その理由は,ある一つの分野のことを極めようと思うなら特殊目的高校に行けば良いからである.
このことから,特性化高校に進学するこどもは,一般的な高校の授業に追いつくこともできず,ただ単に高校卒業の資格が欲しいこどもが集まる
学校と認識されており,「たくさん勉強しなくてもよい学校」と揶揄されることが多い.
\section{日本との比較}
このsectionでは,次の2つのことを比較したいと思う.(ア)「中学校から高校への進学率及び高校から大学への進学率」,
(イ)「公私の高校及び大学の授業料(1年当たり)」\\
 (ア):\underline{中学校から高校への進学率及び高校から大学への進学率}\\
 \begin{center}
    \begin{tabular}{|c|c|c|}
    \hline
    {}  & 日本 & 韓国\\
    \hline 
    中学から高校 & 99% & 99%\\
    \hline
    高校から大学 & 60% & 75%\\
    \hline

\end{tabular}
 \end{center}
中学から高校への進学率は日本と全く変わらない.だが,高校から大学への進学率は韓国のほうが15%程度高いのが分かる.ここで,短大を含む大学
の数を比較してみる.日本は,約1090校である.韓国は,426校である.なぜ,日本の方が大学数は多いのに,韓国の方が進学率が高いのだろうか.
これについては,私は次のような理由があるのではないかと思う.\\
 まず,韓国は日本よりも学歴が重視される社会構造である.我が国日本では,もちろん学歴を重視する職業も多くあるし,一方,そこまで重視しない職業もたくさん
ある.なので,日本の方が多様な進路が存在するという点で,大学進学率はそこまで高くないのではないだろうか.
更に韓国においては,大学を卒業していないと正社員で就職することが難しくなってしまうことが多くあるという.そのため,とりあえず大学に行っておく
という気持ちで行く人が多いためこのような進学率になっているのではないだろうか.\\
 次に,前段の理由と少し重なる部分があるが,日本では4年生大学以外の進路が充実しているということである.例えば,専門学校や短期大学,大学校などである.
そのため韓国のほうが大学への進学率が高いのではないかと考えられる.\\
 (イ):\underline{公私の高校及び大学の授業料(1年当たり)}\\
 \begin{center}
    \begin{tabular}{|c|c|c|}
        \hline
        {} & 日本 & 韓国\\
        \hline
        公立高校 & 512,971円 & 713000ウォン(=76,452円)\\
        \hline
        私立高校 & 1,054,444円 & 13,358,000ウォン(=1,432,323円)\\
        \hline
        国公立大学 & 535,800円 & 4,195,000ウォン(=449,812円)\\
        \hline
        私立大学 & 1,477,339円 & 7,426,000ウォン(=796,259円)\\
        \hline
    \end{tabular}
\end{center}



\end{document}